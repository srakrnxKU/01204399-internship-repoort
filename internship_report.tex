% !TEX program = xelatex

\documentclass{internshipreport}[a4]

\usepackage{xltxtra} 
\XeTeXlinebreaklocale "th_TH"
\usepackage{fontspec}
\defaultfontfeatures{Mapping=tex-text,Scale=MatchLowercase}
\setmainfont{TH Sarabun New}
\setmonofont{Tlwg Typist}

\title{การเขียนรายงานฝึกงานที่เขียนเทมเพลทนานกว่ารายงาน}
\entitle{Writing a nonsensible internship report just to test the template}
\location{ครัวป้าวง ป่ายุบใน}
\date{2562}
\author{ศิระกร ลำใย}
\studentid{5910500023}

\begin{document}

\maketitle

\section*{บทคัดย่อ}

\textit{ข้อความมั่วๆ ไม่มีแก่นสารอะไร}

"แม่มดตายแล้ว" -- Thatcher, ชาร์ทเพลง และการตัดสินใจของ BBC

ในปี 2013 Margaret Thatcher อดีตนายกรัฐมนตรีของอังกฤษ ผู้ซึ่งยึดถืออุดมการณ์เสรีนิยมใหม่ (neoliberalism) ได้เสียชีวิตลง ปฏิกิริยาของกลุ่มผู้ต่อต้านพรรคอนุรักษ์นิยม (Conservative) ที่ Thatcher เคยสังกัดนั้นค่อนข้างรุนแรง

กลุ่มผู้ต่อต้านที่กล่าวถึงข้างต้น สร้างแคมเปญเพื่อดันเพลง "Ding-Dong, The Witch Is Dead" (ติ๊งต่อง แม่มดตายแล้ว) ให้ขึ้นชาร์ทเพลงของอังกฤษ แน่นอนว่า "แม่มด" ในเพลงนี้เป็นใครไม่ได้อีกนอกจากตัวของ Thatcher เอง เพลงนี้ไต่ขึ้น "The Official Chart" ของสหราชอาณาจักรไปในอันดับที่สอง และเป็นธรรมดาของการจัดอันดับชาร์ทเพลงที่จะต้องมีการออกอากาศเพลงที่ติดชาร์ท

คำถามคือบีบีซี ในฐานะ "สื่อ" จะรักษาซึ่งจุดยืนของความเหมาะสมระหว่างการเคารพผู้เสียชีวิต กับการไม่เลือกเซนเซอร์เนื้อหาไว้ได้อย่างไร

\section{บทคัดย่อ}

Tony Hall ผู้อำนวยการใหญ่ของ BBC ในขณะนั้น ตัดสินใจไม่แบนเพลงนี้ และยกอำนาจเพิ่มเติมให้ Ben Cooper ซึ่งคุมช่องวิทยุ Radio One ที่ปกติจะออกอากาศรายการชาร์ทเพลงดังกล่าว ในการตัดสินใจเพิ่มเติม

Ben Cooper ได้เขียนบล็อกบรรยายถึงการตัดสินใจเลือกออกหรือไม่ออกอากาศเพลงนี้ โดยแสดงมุมมองส่วนตัวของเขาซึ่งมองแคมเปญดังกล่าวว่า "ไร้รสนิยม" (distasteful) อย่างไรก็ตามเพลงทุกเพลงที่ขึ้นชาร์ทนั้นสร้างประวัติศาสตร์ที่ตัวเขาไม่สามารถมองข้ามไปได้เช่นกัน

ผลสรุปของการออกอากาศชาร์ทเพลงในครั้งนั้น คือ Radio One เลือกที่จะออกอากาศเพลงยาว 5 วินาทีจากเพลงเต็ม 51 วินาที พร้อมกับบทบรรยายสั้นๆ ถึงที่มาของการติดชาร์ท และมีแม้แต่กระทั่งการอ้างถึงข้อวิพากษ์ที่มีต่อ Thatcher เอง

ในฐานะสื่อที่ต้องธำรงซึ่งความเป็นกลางและรักษาความสมดุลของการรายงานความจริงและการให้เกียรติ กรณีของ Thatcher และเพลงดังกล่าวน่าจะเป็นตัวอย่างที่ดี และแม้แต่ในการวิพากษ์สถานการณ์เร็วๆ นี้ การยึดมั่นในจุดยืนพร้อมกันโดยยังแสดงความสุภาพของตนเองก็ยังเป็นสิ่งที่เป็นไปได้อยู่


\end{document}